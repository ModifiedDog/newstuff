% Options for packages loaded elsewhere
\PassOptionsToPackage{unicode}{hyperref}
\PassOptionsToPackage{hyphens}{url}
%
\documentclass[
]{article}
\usepackage{lmodern}
\usepackage{amssymb,amsmath}
\usepackage{ifxetex,ifluatex}
\ifnum 0\ifxetex 1\fi\ifluatex 1\fi=0 % if pdftex
  \usepackage[T1]{fontenc}
  \usepackage[utf8]{inputenc}
  \usepackage{textcomp} % provide euro and other symbols
\else % if luatex or xetex
  \usepackage{unicode-math}
  \defaultfontfeatures{Scale=MatchLowercase}
  \defaultfontfeatures[\rmfamily]{Ligatures=TeX,Scale=1}
\fi
% Use upquote if available, for straight quotes in verbatim environments
\IfFileExists{upquote.sty}{\usepackage{upquote}}{}
\IfFileExists{microtype.sty}{% use microtype if available
  \usepackage[]{microtype}
  \UseMicrotypeSet[protrusion]{basicmath} % disable protrusion for tt fonts
}{}
\makeatletter
\@ifundefined{KOMAClassName}{% if non-KOMA class
  \IfFileExists{parskip.sty}{%
    \usepackage{parskip}
  }{% else
    \setlength{\parindent}{0pt}
    \setlength{\parskip}{6pt plus 2pt minus 1pt}}
}{% if KOMA class
  \KOMAoptions{parskip=half}}
\makeatother
\usepackage{xcolor}
\IfFileExists{xurl.sty}{\usepackage{xurl}}{} % add URL line breaks if available
\IfFileExists{bookmark.sty}{\usepackage{bookmark}}{\usepackage{hyperref}}
\hypersetup{
  pdftitle={Categorical Data Analysis},
  pdfauthor={Andy Grogan-Kaylor},
  hidelinks,
  pdfcreator={LaTeX via pandoc}}
\urlstyle{same} % disable monospaced font for URLs
\usepackage[margin=1in]{geometry}
\usepackage{color}
\usepackage{fancyvrb}
\newcommand{\VerbBar}{|}
\newcommand{\VERB}{\Verb[commandchars=\\\{\}]}
\DefineVerbatimEnvironment{Highlighting}{Verbatim}{commandchars=\\\{\}}
% Add ',fontsize=\small' for more characters per line
\usepackage{framed}
\definecolor{shadecolor}{RGB}{248,248,248}
\newenvironment{Shaded}{\begin{snugshade}}{\end{snugshade}}
\newcommand{\AlertTok}[1]{\textcolor[rgb]{0.94,0.16,0.16}{#1}}
\newcommand{\AnnotationTok}[1]{\textcolor[rgb]{0.56,0.35,0.01}{\textbf{\textit{#1}}}}
\newcommand{\AttributeTok}[1]{\textcolor[rgb]{0.77,0.63,0.00}{#1}}
\newcommand{\BaseNTok}[1]{\textcolor[rgb]{0.00,0.00,0.81}{#1}}
\newcommand{\BuiltInTok}[1]{#1}
\newcommand{\CharTok}[1]{\textcolor[rgb]{0.31,0.60,0.02}{#1}}
\newcommand{\CommentTok}[1]{\textcolor[rgb]{0.56,0.35,0.01}{\textit{#1}}}
\newcommand{\CommentVarTok}[1]{\textcolor[rgb]{0.56,0.35,0.01}{\textbf{\textit{#1}}}}
\newcommand{\ConstantTok}[1]{\textcolor[rgb]{0.00,0.00,0.00}{#1}}
\newcommand{\ControlFlowTok}[1]{\textcolor[rgb]{0.13,0.29,0.53}{\textbf{#1}}}
\newcommand{\DataTypeTok}[1]{\textcolor[rgb]{0.13,0.29,0.53}{#1}}
\newcommand{\DecValTok}[1]{\textcolor[rgb]{0.00,0.00,0.81}{#1}}
\newcommand{\DocumentationTok}[1]{\textcolor[rgb]{0.56,0.35,0.01}{\textbf{\textit{#1}}}}
\newcommand{\ErrorTok}[1]{\textcolor[rgb]{0.64,0.00,0.00}{\textbf{#1}}}
\newcommand{\ExtensionTok}[1]{#1}
\newcommand{\FloatTok}[1]{\textcolor[rgb]{0.00,0.00,0.81}{#1}}
\newcommand{\FunctionTok}[1]{\textcolor[rgb]{0.00,0.00,0.00}{#1}}
\newcommand{\ImportTok}[1]{#1}
\newcommand{\InformationTok}[1]{\textcolor[rgb]{0.56,0.35,0.01}{\textbf{\textit{#1}}}}
\newcommand{\KeywordTok}[1]{\textcolor[rgb]{0.13,0.29,0.53}{\textbf{#1}}}
\newcommand{\NormalTok}[1]{#1}
\newcommand{\OperatorTok}[1]{\textcolor[rgb]{0.81,0.36,0.00}{\textbf{#1}}}
\newcommand{\OtherTok}[1]{\textcolor[rgb]{0.56,0.35,0.01}{#1}}
\newcommand{\PreprocessorTok}[1]{\textcolor[rgb]{0.56,0.35,0.01}{\textit{#1}}}
\newcommand{\RegionMarkerTok}[1]{#1}
\newcommand{\SpecialCharTok}[1]{\textcolor[rgb]{0.00,0.00,0.00}{#1}}
\newcommand{\SpecialStringTok}[1]{\textcolor[rgb]{0.31,0.60,0.02}{#1}}
\newcommand{\StringTok}[1]{\textcolor[rgb]{0.31,0.60,0.02}{#1}}
\newcommand{\VariableTok}[1]{\textcolor[rgb]{0.00,0.00,0.00}{#1}}
\newcommand{\VerbatimStringTok}[1]{\textcolor[rgb]{0.31,0.60,0.02}{#1}}
\newcommand{\WarningTok}[1]{\textcolor[rgb]{0.56,0.35,0.01}{\textbf{\textit{#1}}}}
\usepackage{longtable,booktabs}
% Correct order of tables after \paragraph or \subparagraph
\usepackage{etoolbox}
\makeatletter
\patchcmd\longtable{\par}{\if@noskipsec\mbox{}\fi\par}{}{}
\makeatother
% Allow footnotes in longtable head/foot
\IfFileExists{footnotehyper.sty}{\usepackage{footnotehyper}}{\usepackage{footnote}}
\makesavenoteenv{longtable}
\usepackage{graphicx,grffile}
\makeatletter
\def\maxwidth{\ifdim\Gin@nat@width>\linewidth\linewidth\else\Gin@nat@width\fi}
\def\maxheight{\ifdim\Gin@nat@height>\textheight\textheight\else\Gin@nat@height\fi}
\makeatother
% Scale images if necessary, so that they will not overflow the page
% margins by default, and it is still possible to overwrite the defaults
% using explicit options in \includegraphics[width, height, ...]{}
\setkeys{Gin}{width=\maxwidth,height=\maxheight,keepaspectratio}
% Set default figure placement to htbp
\makeatletter
\def\fps@figure{htbp}
\makeatother
\setlength{\emergencystretch}{3em} % prevent overfull lines
\providecommand{\tightlist}{%
  \setlength{\itemsep}{0pt}\setlength{\parskip}{0pt}}
\setcounter{secnumdepth}{5}

\title{Categorical Data Analysis}
\author{Andy Grogan-Kaylor}
\date{2020-07-29}

\begin{document}
\maketitle

{
\setcounter{tocdepth}{2}
\tableofcontents
}
\hypertarget{contingency-tables}{%
\section{Contingency Tables}\label{contingency-tables}}

\begin{Shaded}
\begin{Highlighting}[]
\NormalTok{htmltools}\OperatorTok{::}\KeywordTok{includeHTML}\NormalTok{(}\StringTok{"../contingency-tables/contingency-tables.html"}\NormalTok{)}
\end{Highlighting}
\end{Shaded}

\textless!DOCTYPE html\textgreater{}

Contingency Tables

Contingency Tables

Andy Grogan-Kaylor

23 May 2020

Key Concepts and Commands

Matrices of data

Probabilities, risks, and odds

{\(\chi^2\)} Tests

tabulate x y, row col chi2

Flipping Two Coins

Setup

Good value labels are key here.

Crosstabulation

Graphing (Mosaic Plot)

Mosaic Plot

Bar Chart

Does a bar chart work to visualize these relationships?

Bar Chart 1

Bar Chart (2)

Option asyvars adds a crucial color element.

Bar Chart 2

Horizontal Bar Chart

And hbar may improve legibility even more.

Bar Chart 3

1961 French Skiiers

Define Matrix

Theme Music

Try Making a Data Set From Matrix

Enter Data By Hand

There are many alternative commands to do this, but the easiest way is using edit.

I have already done this. Note the structure of the data is different from above.

Mosaic Plot

Mosaic Plot Attempt 1

Mosaic Plot (2)

Mosaic Plot Attempt 2

Definitions and Notation

Counts

{\(\begin{matrix} c_{ij} &amp; c_{ij} &amp; c_{i\bullet} \\ c_{ij} &amp; c_{ij} &amp; c_{i\bullet} \\ c_{\bullet j} &amp; c_{\bullet j} &amp; c_{\bullet \bullet} \\ \end{matrix}\)}

Probabilities

{\(\begin{matrix}p_{ij} &amp; p_{ij} &amp; p_{i\bullet} \\ p_{ij} &amp; p_{ij} &amp; p_{i \bullet} \\ p_{\bullet j} &amp; p_{\bullet j} &amp; p_{\bullet \bullet} \\ \end{matrix}\)}

Terms

{\(p_{ij}\)} are joint probabilities.

{\(p_{i \bullet}\)} and {\(p_{\bullet j}\)} are marginal probabilities.

{\(p_{ij} \mid p_{i \bullet}\)} and {\(p_{ij} \mid p_{\bullet j}\)} are conditional probabilities.

Formulas

Counts

{\(\sum_{1}^{i} \sum_{1}^{j} c_{ij} = N\)}

Probabilities

{\(\sum_{1}^{i} \sum_{1}^{j} p_{ij} = 1.0\)}

Expected Probabilities {\(p\)} and Counts {\(m\)} or Frequencies

{\(p_{ij} = p_{i \bullet} p_{\bullet j}\)}

{\(m_{ij} = \frac{m_{i \bullet} m_{\bullet j}}{m_{\bullet \bullet}}\)}

Observed counts are represented by {\(c\)} while expected counts are represented by {\(m\)}.

Fundamental Rule

{\[\text{conditional = joint / marginal}\]}

Independence (Robert Mare)

If independence is true, then joint probabilities = products of marginal probabilities.

That is, under independence, the conditional distribution equals the marginal distribution.

Under independence, row membership provides no information about the column distribution; and column membership provides no information about the row distribution.

Independence is a model, which is never exactly true in the real world.

Observed vs.~Expected

Chi-Square Test

{\(\chi^2 = \Sigma \frac{(O-E)^2}{E}\)}

Compare With Tabulate

Risk Differences and Risk Ratios (Relative Risk)

Following Viera, 2008:

{\(\begin{bmatrix}a &amp; b \\ c &amp; d\end{bmatrix}\)}

Develop Outcome

Do Not Develop Outcome

Exposed

a

b

Not Exposed

c

d

{\(R = \frac{a}{a+b}\)} (in Exposed)

{\(RR =\frac{\text{risk in exposed}}{\text{risk in not exposed}} = \frac{a/(a+b)}{c/(c+d)}\)}

Odds Ratios

Develop Outcome

Do Not Develop Outcome

Exposed

a

b

Not Exposed

c

d

{\(OR =\)}

{\(\frac{\text{odds that exposed person develops outcome}}{\text{odds that unexposed person develops outcome}}\)}

{\(= \frac{\frac{a}{a+b} / \frac{b}{a+b}}{\frac{c}{c+d} / \frac{d}{c+d}} = \frac{a/b}{c/d} = \frac{ad}{bc}\)}

Properties of the Odds Ratio (Robert Mare)

In general for the 2 X 2 Table,

{\(0 &lt; OR &lt; 1\)}

indicates that one row is less likely to make the first response than the other row.

{\(1 &lt; OR &lt; \infty\)}

indicates that one row is more likely to make the first response than the other row.

\end{document}
